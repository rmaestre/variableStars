\documentclass[]{article}
\usepackage{lmodern}
\usepackage{amssymb,amsmath}
\usepackage{ifxetex,ifluatex}
\usepackage{fixltx2e} % provides \textsubscript
\ifnum 0\ifxetex 1\fi\ifluatex 1\fi=0 % if pdftex
  \usepackage[T1]{fontenc}
  \usepackage[utf8]{inputenc}
\else % if luatex or xelatex
  \ifxetex
    \usepackage{mathspec}
  \else
    \usepackage{fontspec}
  \fi
  \defaultfontfeatures{Ligatures=TeX,Scale=MatchLowercase}
\fi
% use upquote if available, for straight quotes in verbatim environments
\IfFileExists{upquote.sty}{\usepackage{upquote}}{}
% use microtype if available
\IfFileExists{microtype.sty}{%
\usepackage{microtype}
\UseMicrotypeSet[protrusion]{basicmath} % disable protrusion for tt fonts
}{}
\usepackage[margin=1in]{geometry}
\usepackage{hyperref}
\hypersetup{unicode=true,
            pdftitle={Experiment on HD174936},
            pdfauthor={Roberto Maestre},
            pdfborder={0 0 0},
            breaklinks=true}
\urlstyle{same}  % don't use monospace font for urls
\usepackage{color}
\usepackage{fancyvrb}
\newcommand{\VerbBar}{|}
\newcommand{\VERB}{\Verb[commandchars=\\\{\}]}
\DefineVerbatimEnvironment{Highlighting}{Verbatim}{commandchars=\\\{\}}
% Add ',fontsize=\small' for more characters per line
\usepackage{framed}
\definecolor{shadecolor}{RGB}{248,248,248}
\newenvironment{Shaded}{\begin{snugshade}}{\end{snugshade}}
\newcommand{\KeywordTok}[1]{\textcolor[rgb]{0.13,0.29,0.53}{\textbf{#1}}}
\newcommand{\DataTypeTok}[1]{\textcolor[rgb]{0.13,0.29,0.53}{#1}}
\newcommand{\DecValTok}[1]{\textcolor[rgb]{0.00,0.00,0.81}{#1}}
\newcommand{\BaseNTok}[1]{\textcolor[rgb]{0.00,0.00,0.81}{#1}}
\newcommand{\FloatTok}[1]{\textcolor[rgb]{0.00,0.00,0.81}{#1}}
\newcommand{\ConstantTok}[1]{\textcolor[rgb]{0.00,0.00,0.00}{#1}}
\newcommand{\CharTok}[1]{\textcolor[rgb]{0.31,0.60,0.02}{#1}}
\newcommand{\SpecialCharTok}[1]{\textcolor[rgb]{0.00,0.00,0.00}{#1}}
\newcommand{\StringTok}[1]{\textcolor[rgb]{0.31,0.60,0.02}{#1}}
\newcommand{\VerbatimStringTok}[1]{\textcolor[rgb]{0.31,0.60,0.02}{#1}}
\newcommand{\SpecialStringTok}[1]{\textcolor[rgb]{0.31,0.60,0.02}{#1}}
\newcommand{\ImportTok}[1]{#1}
\newcommand{\CommentTok}[1]{\textcolor[rgb]{0.56,0.35,0.01}{\textit{#1}}}
\newcommand{\DocumentationTok}[1]{\textcolor[rgb]{0.56,0.35,0.01}{\textbf{\textit{#1}}}}
\newcommand{\AnnotationTok}[1]{\textcolor[rgb]{0.56,0.35,0.01}{\textbf{\textit{#1}}}}
\newcommand{\CommentVarTok}[1]{\textcolor[rgb]{0.56,0.35,0.01}{\textbf{\textit{#1}}}}
\newcommand{\OtherTok}[1]{\textcolor[rgb]{0.56,0.35,0.01}{#1}}
\newcommand{\FunctionTok}[1]{\textcolor[rgb]{0.00,0.00,0.00}{#1}}
\newcommand{\VariableTok}[1]{\textcolor[rgb]{0.00,0.00,0.00}{#1}}
\newcommand{\ControlFlowTok}[1]{\textcolor[rgb]{0.13,0.29,0.53}{\textbf{#1}}}
\newcommand{\OperatorTok}[1]{\textcolor[rgb]{0.81,0.36,0.00}{\textbf{#1}}}
\newcommand{\BuiltInTok}[1]{#1}
\newcommand{\ExtensionTok}[1]{#1}
\newcommand{\PreprocessorTok}[1]{\textcolor[rgb]{0.56,0.35,0.01}{\textit{#1}}}
\newcommand{\AttributeTok}[1]{\textcolor[rgb]{0.77,0.63,0.00}{#1}}
\newcommand{\RegionMarkerTok}[1]{#1}
\newcommand{\InformationTok}[1]{\textcolor[rgb]{0.56,0.35,0.01}{\textbf{\textit{#1}}}}
\newcommand{\WarningTok}[1]{\textcolor[rgb]{0.56,0.35,0.01}{\textbf{\textit{#1}}}}
\newcommand{\AlertTok}[1]{\textcolor[rgb]{0.94,0.16,0.16}{#1}}
\newcommand{\ErrorTok}[1]{\textcolor[rgb]{0.64,0.00,0.00}{\textbf{#1}}}
\newcommand{\NormalTok}[1]{#1}
\usepackage{graphicx,grffile}
\makeatletter
\def\maxwidth{\ifdim\Gin@nat@width>\linewidth\linewidth\else\Gin@nat@width\fi}
\def\maxheight{\ifdim\Gin@nat@height>\textheight\textheight\else\Gin@nat@height\fi}
\makeatother
% Scale images if necessary, so that they will not overflow the page
% margins by default, and it is still possible to overwrite the defaults
% using explicit options in \includegraphics[width, height, ...]{}
\setkeys{Gin}{width=\maxwidth,height=\maxheight,keepaspectratio}
\IfFileExists{parskip.sty}{%
\usepackage{parskip}
}{% else
\setlength{\parindent}{0pt}
\setlength{\parskip}{6pt plus 2pt minus 1pt}
}
\setlength{\emergencystretch}{3em}  % prevent overfull lines
\providecommand{\tightlist}{%
  \setlength{\itemsep}{0pt}\setlength{\parskip}{0pt}}
\setcounter{secnumdepth}{0}
% Redefines (sub)paragraphs to behave more like sections
\ifx\paragraph\undefined\else
\let\oldparagraph\paragraph
\renewcommand{\paragraph}[1]{\oldparagraph{#1}\mbox{}}
\fi
\ifx\subparagraph\undefined\else
\let\oldsubparagraph\subparagraph
\renewcommand{\subparagraph}[1]{\oldsubparagraph{#1}\mbox{}}
\fi

%%% Use protect on footnotes to avoid problems with footnotes in titles
\let\rmarkdownfootnote\footnote%
\def\footnote{\protect\rmarkdownfootnote}

%%% Change title format to be more compact
\usepackage{titling}

% Create subtitle command for use in maketitle
\newcommand{\subtitle}[1]{
  \posttitle{
    \begin{center}\large#1\end{center}
    }
}

\setlength{\droptitle}{-2em}

  \title{Experiment on HD174936}
    \pretitle{\vspace{\droptitle}\centering\huge}
  \posttitle{\par}
    \author{Roberto Maestre}
    \preauthor{\centering\large\emph}
  \postauthor{\par}
      \predate{\centering\large\emph}
  \postdate{\par}
    \date{10/24/2018}


\begin{document}
\maketitle

\href{https://travis-ci.org/rmaestre/variableStars}{\includegraphics{https://travis-ci.org/rmaestre/variableStars.svg?branch=master}}

Please,
\href{https://github.com/rmaestre/variableStars/blob/master/README.md}{read
the package introduction} for initial details about the software,
algorithms, workflow and data used.

\section{\texorpdfstring{Neural Network approach for detecting
\(\Delta \nu\) and
\(\delta r\)}{Neural Network approach for detecting \textbackslash{}Delta \textbackslash{}nu and \textbackslash{}delta r}}\label{neural-network-approach-for-detecting-delta-nu-and-delta-r}

\href{https://arxiv.org/pdf/1610.02357.pdf}{Depthwise separable
convolutions} for machine learning consists in \emph{spatial convolution
performed independently over each channel of an input, followed by a
pointwise convolution, projecting the channels output by the depthwise
convolution onto a new channel space. }

The next figure represents an overview of the main Neural Network
architecture used to identify \(\Delta \nu\) and \(\delta r\) from a
given power espectrum of a variable star:

{[}meter imagen{]}

\section{Methodology}\label{methodology}

Each input channel channel is processed with the \emph{variableStars}
package. The data from the variable star is process by the main method

\begin{Shaded}
\begin{Highlighting}[]
\NormalTok{result <-}\StringTok{ }\KeywordTok{process}\NormalTok{(}
\NormalTok{  dt.star}\OperatorTok{$}\NormalTok{frequency,}
\NormalTok{  dt.star}\OperatorTok{$}\NormalTok{amplitude,}
  \DataTypeTok{filter =} \StringTok{"uniform"}\NormalTok{,}
  \DataTypeTok{gRegimen =} \DecValTok{0}\NormalTok{,}
  \DataTypeTok{minDnu =} \DecValTok{15}\NormalTok{,}
  \DataTypeTok{maxDnu =} \DecValTok{95}\NormalTok{,}
  \DataTypeTok{dnuValue =} \OperatorTok{-}\DecValTok{1}\NormalTok{,}
  \DataTypeTok{dnuGuessError =} \DecValTok{10}\NormalTok{,}
  \DataTypeTok{dnuEstimation =} \OtherTok{TRUE}\NormalTok{,}
  \DataTypeTok{numFrequencies =} \DecValTok{30}\NormalTok{,}
  \DataTypeTok{debug =}\NormalTok{ F}
\NormalTok{)}
\end{Highlighting}
\end{Shaded}

and each channel is extracted as:

\begin{itemize}
\tightlist
\item
  Fourier transform
\end{itemize}

\includegraphics{NN-dnudr_files/figure-latex/unnamed-chunk-2-1.pdf}

\begin{itemize}
\tightlist
\item
  Histogram of differences
\end{itemize}

\includegraphics{NN-dnudr_files/figure-latex/unnamed-chunk-3-1.pdf}

\begin{itemize}
\tightlist
\item
  Autocorrelation
\end{itemize}

\includegraphics{NN-dnudr_files/figure-latex/unnamed-chunk-4-1.pdf}

\section{Experiments}\label{experiments}

\subsection{Experiments I}\label{experiments-i}

\subsection{Experiments II}\label{experiments-ii}

\subsection{Experiments III}\label{experiments-iii}


\end{document}
